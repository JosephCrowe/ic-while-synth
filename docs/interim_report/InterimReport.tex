\documentclass[a4paper,twoside,notitlepage]{article}

\usepackage{parskip}
\usepackage{hyperref}
\usepackage{algorithmicx}
\usepackage[numbib]{tocbibind}
\usepackage[fleqn]{amsmath}
\usepackage{comment}
\usepackage{qtree}
\usepackage{float}
\usepackage{fancyvrb}

\floatstyle{boxed}
\restylefloat{figure}

\renewcommand{\bibsection}{\section{\bibname}}

\newcommand{\ttt}{\texttt}
\newcommand{\trm}{\textrm}

\begin{document}

\title{Synthesis of Simple While Programs Using Inductive Logic Programming:
       Interim Report}
\author{Joseph Crowe \\ Supervised by Krysia Broda and Mark Law}
\date{09 March 2015}
\maketitle

\abstract

\tableofcontents
\clearpage

\section{Introduction}

We shall address the problem of automatically writing computer programs in an 
imperative programming language (see section~\ref{sec:outlng}, While Language) 
given a specification of the program's behaviour written in a simple, abstract 
language understandable by users without programming expertise.

In particular, we have examined the notion of allowing the specification to 
consist of a finite set of input/output pairs that the generated program 
should satisfy (see section~\ref{sec:inplng}, Specification Language).

This can be viewed as a machine-learning problem, where the program generator 
\emph{learns} the correct program by considering examples of its behaviour. Since the 
discrete and exacting nature of computer programming seems to lend itself well to a 
logic-based learning approach, we ask whether the paradigm of Inductive Logic 
Programming (ILP)\cite{muggleton94} is well-suited to solve this problem. More 
specifically, our working implementation consists of an Answer Set Program run using 
tools from the Potsdam Answer Set Solving Collection\cite{potassco} (see 
section~\ref{sec:asp}, Answer Set Programming).

As a simple example, the following specification:

\begin{tabular}{ll}
    \textbf{Input} & \textbf{Output}
\\  $x = 0$        & $y = 0$
\\  $x = 1$        & $y = 2$
\\  $x = 2$        & $y = 4$
\\  $x = 3$        & $y = 6$
\end{tabular}

is one that many programs could satisfy (there is an $n$-degree polynomial for each 
integer $n \geq 4$ which interpolates these points), but the program which the user 
probably has in mind is:

\begin{algorithmic}[H]
\State $y \gets 2 \times x$
\end{algorithmic}

This could be (informally) verified by testing the program against additional 
examples not used to generate the program, or perhaps formally verified using 
automated proof techniques, if the program's specification were available in a 
suitable form.

This has possible applications in Computing education: a teacher with students who are 
learning how to program may wish to generate \emph{multiple} different programs, one 
for each student, to be used by the student in an exercise by extending, analysing the 
behaviour of, etc. If the teacher is able to quickly generate these multiple programs 
without spending the time to write each one, this task becomes much more scalable in 
the number of students.

Another application is in data processing software tools (such as spreadsheets) whose 
users are likely to be non-programmers, but whose usefulness could be greatly increased 
by the automation of complex data processing tasks, representable as computer programs, 
which could be specified by the user as a list of input/output examples.

In previous work, this problem has been approached in many different ways, and 
several systems have been built which are capable of synthesising programs in 
this manner (see section~\ref{sec:prvsyn}, Previous Work in Program 
Synthesis), but the problem has not in its full generality been resolved. We 
hope to gain an insight into this by approaching it with new tools and 
techniques.

\section{Background}

\subsection{Previous Work in Program Synthesis} \label{sec:prvsyn}

Saurabh Srivastava, Sumit Gulwani, et al.\ in 2010 showed\cite{popl10syn} 
that, given a functional specification of an imperative program, relating its 
inputs and outputs with logical formulae, and given also the looping structure 
of the the program, it is possible to infer a program satisfying these 
constraints, using techniques previously used for program verification.

David Perelman, Sumit Gulwani, et al.\ in 2014 developed a system\cite{tds} 
based on the principle of ``programming by example,'' which generates and 
iteratively refines a program from a sequence of increasingly more general 
sets of input/output examples. Notably, their system is not limited to 
imperative programs, but allows arbitrary domain-specific languages to be 
defined and used with the system.

\subsection{Answer Set Programming} \label{sec:asp}

Answer Set Programming (ASP) is a form of logic programming particularly 
suited to search problems over large domains (such as the domain of all 
imperative programs of a given length). An Answer Set Program distinguishes 
the result to be computed by a set of propositional facts and rules in the 
domain. Importantly, ASP includes \emph{aggregates}, \emph{integrity 
constraints}, and \emph{negation as failure}, where a rule can be conditional 
on the inability of prove that some proposition is true, which allows 
\emph{non-monotonic reasoning}. The result of running an ASP is an 
\emph{answer set} of facts which is consistent with the program, forms a 
\emph{stable model}, and is minimal in a certain sense. For further details, 
see \cite{glimpse}, which gives a concise introduction to the paradigm.

In 2013, Domenico Corapi et al.\ developed ASPAL\cite{aspal}, a system written 
in ASP for solving classical Inductive Logic Programming tasks, which brings 
to bear against these problems the efficiency of existing ASP solvers, and 
allows their optimisation features to be used to direct the search towards a 
desired optimal solution.

Building on this, in 2014, Mark Law et al.\ in 2014 developed 
ILASP\cite{ilasp}, implementing a form of ILP where an Answer Set Program 
representing the solution is learned inductively from positive and negative 
examples, with applications in agent planning.

\section{Current Work}

We have begun development of a utility, based on the Potassco\cite{potassco} 
tools for Answer Set Programming, which reads a user-written specification 
including a language bias and test examples, and produces imperative programs 
in a simple ``While'' language.

\subsection{While Language} \label{sec:outlng}

The following context-free grammar gives the (abstract) syntax of the While 
language. The basic elements of a program are (at the moment) integer 
constants and named variables. The exact set of possible constants and 
variables is specific to the program being generated, is given by the language 
bias.
\begin{align*}
   \trm{Const} \to\ & ... \mid -2 \mid -1 \mid 0 \mid 1 \mid 2 \mid ...
\\ \trm{Var}   \to\ & a \mid b \mid c \mid ... \mid x \mid y \mid z \mid ...
\end{align*}
A While program is a possible empty sequence of \emph{commands}, where a 
\emph{command} assigns a value to a variable, or takes the form of an \ttt{if} 
or \ttt{while} statement with a boolean guard and a further sequence of 
commands:
\begin{align*}
   \trm{Prog}  \to\ & \trm{Cmds}
\\ \trm{Cmds}  \to\ & \varepsilon \mid \trm{Cmd} \ttt{; } \trm{Cmds}
\\ \trm{Cmd}   \to\ & \trm{Var} \ \ttt{=} \ \trm{Expr}
\\            \mid\ & \ttt{if (} \trm{Bool} \ttt{) \char`\{\ } \trm{Cmds} \ttt{ \char`\}}
\\            \mid\ & \ttt{while (} \trm{Bool} \ttt{) \char`\{\ } \trm{Cmds} \ttt{ \char`\}}
\end{align*}
An arithmetic expression is a sum, difference, product, integer quotient, or 
integer remainder of two constants and/or variables. A boolean guard can test 
the relation less-than ($<$) or less-than-or-equal ($\leq$) on constants and 
variables. The structure of expressions is deliberately kept as simple as 
possible, to reduce the search space of programs:
\begin{align*}
   \trm{Expr}  \to\ & \trm{LExpr}\ (\ \ttt{+} \mid \ttt{-} \mid \ttt{*} \mid
                                      \ttt{/} \mid \ttt{\%} \ )\ \trm{LExpr}
\\ \trm{Bool}  \to\ & \trm{LExpr}\ (\ \ttt{<} \mid \ttt{<=}\ )\ \trm{LExpr}
\\ \trm{LExpr} \to\ & \trm{Const} \mid \trm{Var}
\end{align*}
Input and output is achieved by setting the initial values of certain 
\emph{input variables} before the program runs, and reading the values of 
certain \emph{output variables} after it has terminated.

The semantics of a While program are the same as C programs with the same 
syntax, except:
\begin{enumerate}
    \item Execution starts at the first command of the program, and ends when 
    the last command has been executed. For simplicity, there is currently no 
    other way to halt execution, but we acknowledge that a \ttt{stop} command 
    may be needed for some programs, and may be added in future.
    \item All variables exist in the same global scope throughout the 
    execution of the program.
    \item Except for input variables, the initial value of every variable is 0.
    We change this in the future this so that variables may not be used unless 
    they are explicitly initialised.
    \item The results of arithmetical operations wrap around so that all 
    values remain within the range of possible integer constants. This is done 
    mainly to reduce the complexity of simulating programs using ASP. An 
    alternative would be to cause the program to fail when this happens, which 
    might be more desirable in eliminating spurious programs taking advantage 
    of integer overflow.
    \item Division by zero, and taking the remainder of division by zero,
    cause the program to formally not terminate.
\end{enumerate}

Here is an example of a program satisfying this grammar, which takes an input 
integer $x$ and computes the output integer $z=\trm{Fib}(x)$, the $x$th 
Fibonacci number:
\begin{Verbatim}[samepage=true]
    z = 1;
    while (1 < x) {
        z = x + z;
        y = z - y;
        x = x - 1;
    };
\end{Verbatim}

A parse tree corresponding to this program (with intermediate non-terminals 
omitted when unambiguous) is given in Figure~\ref{fig:fibtree}.

\begin{figure}[h]
  \caption{The parse tree of a While program.}
  \label{fig:fibtree}
  \Tree[ .Prog [ .Cmds
    [ .\ttt{=} \ttt{x} \ttt{1} ]
    [ .Cmds
      [ .\ttt{while}
        [ .Bool [ .\ttt{<} \ttt{1} \ttt{x} ] ]
        [ .Cmds
          [ .\ttt{=} \ttt{z} [ .\ttt{+} \ttt{x} \ttt{z} ] ]
            [ .Cmds
              [ .\ttt{=} \ttt{y} [ .\ttt{-} \ttt{z} \ttt{y} ] ]
              [ .Cmds
                [ .\ttt{=} \ttt{x} [ .\ttt{-} \ttt{x} \ttt{1} ] ]
                $\varepsilon$
              ]
            ]
          ]
        ]
      $\varepsilon$
    ]
  ]]
\end{figure}

\subsection{Program Interpreter}

We have written in ASP an interpreter for the While language which, given 
facts specifying the initial values of input variables and the contents of 
each \emph{line} of the program, includes in the answer set facts saying 
(among other things) whether the program terminated, and if it did, what the 
final values of the output variables are.

The interpreter is capable of running at multiple traces of execution at once, 
each with possibly different input values, so the relevant facts about 
execution are also parameterised by a \emph{run identifier}. This 
functionality is essential to be able to learn programs that satisfy at once 
multiple test examples, as will be seen in section~\ref{sec:prgsyn}, Program 
Synthesiser.

The following facts specify a program which has an input variable $x$, and 
computes $z = \textrm{Fib}(x)$, the $x$th Fibonacci number, and also specifies 
6 different runs to perform:
\begin{Verbatim}[samepage=true]
    #const line_max=6.      % Maximum line number.
    var(x). var(y). var(z). % All variables used.
    
    line_instr(1, set(z, con(1))).               % z = 1;
    line_instr(2, while(lt(con(1), var(x)), 3)). % while (1 < x) {
    line_instr(3, set(z, add(var(y), var(z)))).  %    z = y + z;
    line_instr(4, set(y, sub(var(z), var(y)))).  %    y = z - y;
    line_instr(5, set(x, sub(var(x), con(1)))).  %    x = x - 1;
    line_instr(6, end_while).                    % }
    
    % Run the program with 6 different initial values of x:
    in(r1,x,1). in(r2,x,2). in(r3,x,3).
    in(r4,x,4). in(r5,x,5). in(r6,x,6).
\end{Verbatim}
The second parameter of the function symbol \texttt{while/2} gives the 
length of its body, i.e.\ the number of instructions it contains, 
excluding the \ttt{while} and \ttt{end\_while} lines.

When run with the While interpreter, this produces an answer set including the 
following facts:
\begin{Verbatim}[samepage=true]
    run_var_out(r1,x,1) run_var_out(r1,y,0) run_var_out(r1,z,1) 
    run_var_out(r2,x,1) run_var_out(r2,y,1) run_var_out(r2,z,1) 
    run_var_out(r3,x,1) run_var_out(r3,y,1) run_var_out(r3,z,2) 
    run_var_out(r4,x,1) run_var_out(r4,y,2) run_var_out(r4,z,3)
    run_var_out(r5,x,1) run_var_out(r5,y,3) run_var_out(r5,z,5)
    run_var_out(r6,x,1) run_var_out(r6,y,5) run_var_out(r6,z,8) 
\end{Verbatim}
The first column gives the final value of $x$ for each run and is not 
interesting, as this is just always 1 (or less), by the loop's guard. The 
second and third columns gives the final values of $y$ and $z$, and we can see 
that, as we hoped, $z=\textrm{Fib}(x_0)$ and $y=\textrm{Fib}(x_0-1)$, where 
$x_0$ is the initial value of $x$ for run $r_{x_0}$.

\subsection{Program Synthesiser} \label{sec:prgsyn}

The example in the previous section shows how the program interpreter 
works, but it is not how the system is intended to be used. In actuality, 
the user specifies what the output values should be for various inputs, 
and the solver abducts a program satisfying those constraints.

This is achieved by including an aggregate defining the domain of 
admissable programs, and integrity constraints to enforce the test 
examples:
\begin{Verbatim}[samepage=true]
    % Exactly one instruction per line.
    1{ line_instr(L, I) : valid_line_instr(L, I) }1 :- line(L).
    
    valid_line_instr(L, set(V, E))      :- line(L), write_var(V), expr(E).
    valid_line_instr(L, if(B, 1..M))    :- line(L), bool(B), M=line_max-L.
    valid_line_instr(L, while(B, 1..M)) :- line(L), bool(B), M=line_max-L-1.
    valid_line_instr(L, end_while)      :- line(L).
    
    % All test examples must halt with the correct value.
    :- run_does_not_halt(R).
    :- run_var_out(R,X,Actual), out(R,X,Expected), Actual != Expected.
\end{Verbatim}
The predicate \ttt{out/3} is the counterpart to \ttt{in/3}, which was 
earlier used to specify the values of input variables, that asserts what 
the final value of a certain variable must be. The predicate 
\ttt{run\_does\_not\_halt/1} holds when a given run exceeds the time 
limit, or for some other reason does not formally terminate (e.g.\ when 
division by zero occurs). Further integrity constraints are included that 
ensure the generated program has a sensible structure, but for the sake of 
brevity are not shown here.

If we fix some facts in addition to the main program that includes the 
above machinery, we can immediately synthesise some simple programs:
\begin{Verbatim}[samepage=true]
    #const line_max=1.  % Number of lines.
    #const int_max=10.  % Maximum integer value.

    % Allowed explicit constants, variables, and writable variables:
    con(1). con(2). var(x). var(y). write_var(y).
    
    % Test examples:
    in(r0,x,0). out(r0,y,0).
    in(r1,x,1). out(r1,y,2).
    in(r2,x,2). out(r2,y,4).
    in(r3,x,3). out(r3,y,6).
\end{Verbatim}
The answer set of which includes:
\begin{Verbatim}[samepage=true]
    line_instr(1,set(y,add(var(x),var(x))))
\end{Verbatim}
Which is the program we expected, $y \gets x + x$.

As a slightly less trivial example, we can make the following 
specification of a program that computes $y = \sum_{i=0}^x i$, the $x$th 
triangular number:
\begin{Verbatim}[samepage=true]
    #const line_max=4.
    #const int_max=11. 
    con(1). var(x). var(y). write_var(x). write_var(y).
    
    in(r0,x,0). out(r0,y,0).
    in(r1,x,1). out(r1,y,1).
    in(r2,x,2). out(r2,y,3).
    in(r3,x,3). out(r3,y,6).
    in(r4,x,4). out(r4,y,10).
\end{Verbatim}
This program is solved in 12 seconds on a 3.4GHz machine, yielding:
\begin{Verbatim}[samepage=true]
    line_instr(1,while(le(con(1),var(x)),2)) % while (1 <= x) {
    line_instr(2,set(y,add(var(x),var(y))))  %    y = x + y
    line_instr(3,set(x,sub(var(x),con(1))))  %    x = x - 1
    line_instr(4,end_while)                  % }
\end{Verbatim}
Which is easily verified to generalise the examples in the intended way.

\subsection{Specification Language} \label{sec:inplng}
For longer and more complicated programs, it is currently necessary to cut 
down the search space by enforcing stronger constraints on the structure 
or contents of the programs. These are currently specified in the same 
syntax as the ASP program, but it would not be difficult to devise a more 
user-friendly format.

To make it easier to interpret the output of the system, we have a utility 
that converts relevant parts of the answer set into a more traditional 
human-readable listing of the program it represents. Another utility 
automates an incremental search for the smallest possible program, by 
first trying to find a program of 1 line, then incrementing the number of 
lines after each failure until a solution is found.

Following are some examples of inputs to the system and their 
corresponding output, illustrating its full functionality.

We can compute the median of three input values:
\begin{Verbatim}[samepage=true]
    #const time_max=line_max+1.
    #const int_max=8.
    var(a). var(b). var(c). var(y). var(z).
    write_var(y). write_var(z).
    
    % Omit from the search space some features we know are not needed:
    while_not_allowed.
    arithmetic_not_allowed.
    
    % Predefine the first three lines of the program, to help the solver:
    preset_line_instr(1, set(y, var(a))).
    preset_line_instr(2, if(lt(var(y), var(b)), 1)).
    preset_line_instr(3, set(y, var(b))).
    
    % Examples given in shorthand: ex(r, a,b,c, z).
    ex(r0, 0,0,0, 0).
    ex(r1, 7,1,1, 1). ex(r5, 1,7,1, 1). ex(r9, 1,1,7, 1).
    ex(r2, 1,7,7, 7). ex(r6, 7,1,7, 7). ex(rA, 7,7,1, 7).
    ex(r3, 1,2,3, 2). ex(r7, 1,3,2, 2). ex(rB, 2,1,3, 2).
    ex(r4, 3,2,1, 2). ex(r8, 2,3,1, 2). ex(rC, 3,1,2, 2).
\end{Verbatim}
This program is solved in 44 seconds on a 3.4GHz machine, yielding:
\begin{Verbatim}[samepage=true]
   1. y = a
   2. if (y < b):
   3.     y = b
   4. if (c < b):
   5.     y = b
   6.     z = a
   7. if (z < c):
   8.     z = c
   9. if (y <= z):
  10.     z = y
\end{Verbatim}
Which can be verified to indeed compute $z = \textrm{median}(a,b,c) = 
\textrm{min}(\textrm{max}(a,b), \textrm{max}(b,c), \textrm{max}(c,a))$.

Finally, the earlier Fibonacci program can be learned from test examples, 
when some help is given with the program structure:
\begin{Verbatim}[samepage=true]
    #const time_max=30.
    #const int_max=10.
    con(1). var(x). var(y). var(z).
    write_var(x). write_var(y). write_var(z).
    
    % Predefine the looping structure of the program,
    % and forbid the solver from generating any further structure:
    preset_line_instr(2, while(lt(con(1), var(x)), line_max-3)).
    preset_line_instr(line_max, end_while).
    
    if_not_allowed. while_not_allowed.
    
    in(r1,x,1). out(r1,z,1).  in(r2,x,2). out(r2,z,1).
    in(r3,x,3). out(r3,z,2).  in(r4,x,4). out(r4,z,3).
    in(r5,x,5). out(r5,z,5).  in(r6,x,6). out(r6,z,8).
\end{Verbatim}
This solves in 113 seconds on a 3.4GHz machine, yielding:
\begin{Verbatim}[samepage=true]
   1. z = 1
   2. while (1 < x):
   3.     z = y + z
   4.     x = x - 1
   5.     y = z - y
   6. end_while
\end{Verbatim}
As expected.

\section{Project Plan} \label{sec:projplan}

Considering that much is still unknown about what the result of the project 
will look like, and that various ideas are being considered and tested, it is 
hard to write down a timetable of what work is going to be done by when. 
However, we can identify several avenues to be explored:
\begin{itemize}

    \item Instead of trying to learn whole programs at once, it has been 
    suggested that we might determine, either manually or automatically, the 
    \emph{midconditions} to be satisfied by the variables at points in 
    execution between the start and end of the program; it would then be 
    possible to split the the program into multiple subprograms with 
    corresponding pre- and post-conditions, and search for these subprograms 
    separately, possibly in parallel.
    
    This would certainly offer a dramatic increase in the speed of the solver 
    and the domain of programs we are able to generate, and if a way were 
    found to generate the midconditions automatically, would provide a very 
    promising direction for the project.

    \item Instead of requiring the user to supply only a finite list of 
    input/output examples, it would be beneficial to accept a functional 
    specification of the computation to be performed by the program, in the 
    form of logical formulae relating the input and output variable values.
    
    It would then be possible to generate as many test examples as needed 
    (limited by the domain of input values) for an iterative approach to 
    generating the program.
    
    Challenges here include defining a representation of the formulae that can 
    express all programs we would want to learn, and extracting test examples 
    from the formulae, if they are not trivially solved equations such as $y = 
    f(x)$.

    \item A limit to the size and complexity of (sub-)programs learned is 
    currently the time taken to solve the ASP. This is strongly related to the 
    size of the \emph{grounding} generated as part of this process, where 
    variables in each rule are substituted with all possible values. The 
    solver could be optimised by reducing the size of this grounding (by 
    inspecting the code and seeing where it could be improved), or by applying 
    other optimisations that eliminate redundant programs from the search 
    space to reduce its size.

\end{itemize}

\section{Evaluation Plan}

As with the Project Plan (see section~\ref{sec:projplan}), it is difficult to 
say exactly what form will be taken by the benchmarks and tests used to 
measure the effectiveness of the system will have developed, but we can 
identify some measurable features that relate to its effectiveness:

\begin{itemize}
    \item Most importantly, perhaps, we must ask whether the generated 
    programs are \emph{correct} in that they exhibit the behaviour that we 
    expected. This could be heuristicically verified by testing particular 
    generated programs against additional test examples, or, perhaps, formally 
    proven for particular programs, using logical pre- and post-conditions.

    \item We can measure the time taken to solve for particular programs, and 
    compare this to existing program-generating systems. Relatedly, we can ask 
    what is the longest or most complex program that our system can generate 
    in a reasonable length of time, or what features in addition to plain 
    arithmetic can be supported in generated programs. Choosing a 
    representative sample of programs for these tests will be important to 
    expose any biases.

    \item We can ask how the number of test examples needed to generate the 
    correct program relates to the program's length or time complexity. It 
    would be expected that longer or more complex programs require more 
    examples, but if the number of examples needed can be minimised, then the 
    system (when used in such a way that the user must provide examples) is 
    more intelligent and less reliant on the user for guidance, and probably 
    also more efficient.
\end{itemize}

\begin{thebibliography}{9}
    \bibitem{muggleton94}
        Muggleton S, De Raedt L (1994).
        \emph{Inductive Logic Programming: Theory and methods}.
        The Journal of Logic Programming, vol. 19-20, page 629-679.
    \bibitem{potassco}
        Potassco, the Potsdam Answer Set Solving Collection. 
        \url{http://potassco.sourceforge.net}
    \bibitem{glimpse}
        Anger C, Konczak K, Linke T, Schaub T (2005).
        \emph{A Glimpse of Answer Set Programming}. 
        \url{http://www.cs.uni-potsdam.de/wv/pdfformat/ankolisc05.pdf}
    \bibitem{popl10syn}
        Srivastava S, Gulwani S, Foster J S (2010).
        \emph{From Program Verification to Program Synthesis}.
        \url{http://www.cs.umd.edu/~saurabhs/pubs/popl10-syn.html}
    \bibitem{tds}
        Perelman D, Gulwani S, Grossman D, Provost P (2014).
        \emph{Test-Driven Synthesis}.
        \url{http://research.microsoft.com/en-us/um/people/sumitg/pubs/pldi14-tds.pdf}
    \bibitem{aspal}
        Corapi D, Russo A, Lupu E (2011)
        \emph{Inductive Logic Programming in Answer Set Programming}. 
        \url{http://ilp11.doc.ic.ac.uk/short_papers/ilp2011_submission_20.pdf}
    \bibitem{ilasp}
        Law M, Russo A, Broda K (2014).
        \emph{Inductive learning of answer set programs}.
        \url{https://www.doc.ic.ac.uk/~ml1909/ILASP_Paper.pdf}
\end{thebibliography}

\end{document}
