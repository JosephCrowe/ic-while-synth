\documentclass[a4paper,twoside,notitlepage,12pt]{article}
\usepackage{parskip}
\usepackage[nottoc,numbib]{tocbibind}
\usepackage{fullpage}
\usepackage{hyperref}
\usepackage{verbatim}
\usepackage{fancyvrb}
\usepackage{tabularx}
\usepackage{ltablex}
\usepackage{amsfonts}
\usepackage{amsmath}

\DefineVerbatimEnvironment{verbatim}{Verbatim}{
    frame=single, baselinestretch=1}

\renewcommand{\bibsection}{\section{\bibname}}
\setcounter{tocdepth}{2}


\begin{document}

\begin{titlepage}
\begin{center}
    {\Large \bfseries Synthesis of Simple While Programs \\ using Answer Set Programming}

    {\large UROP Project Status, 14 July 2015} \\[0.5cm]
    
    {\large Joseph Crowe} \texttt{<jjc311@imperial.ac.uk>}
    
    \url{http://www.doc.ic.ac.uk/~jjc311/while-synth}\\[2cm]

    \tableofcontents
\end{center}
\end{titlepage}

\clearpage
\section{Introduction to the Problem} \label{sec:intro}

The project is concerned with the automatic synthesis of computer programs in a simple 
Turing-complete imperative programming language with integer arithmetic, branching, and
looping, such as the following:

\begin{verbatim}
// Input x, a non-negative integer.
s = 0;
d = 1;
while (d < x): {
    m = x % d;
    if (m < 1) {
        s = s + d;
    }
    d = d + 1;
}
// Output s, the sum of the proper divisors of x.
\end{verbatim}

The details of this language are subject to change as the project progresses, but it is 
conceptually the same as the basic imperative subset common to the C and Python 
programming languages. The syntax and semantics are documented in Chapter 4 of 
\cite{final}, as they were at the time of writing.

The task is to provide a way to automatically synthesise such a program from a 
specification given by the user, consisting of:

\begin{enumerate}
    \item A \emph{language bias} saying what syntactic elements are allowed to occur in 
    the synthesised program: for example, what integer constants and arithmetic 
    operators may be used, or how many \verb|if| and \verb|while| statements may 
    occur, and how they may be structured; and \textbf{one or both} of:

    \item A \emph{list of examples}, saying exactly what the output of the program 
    should be when certain variables are initialised with certain input values. For 
    example, the following, which specifies by example the above 
    sum-of-divisors program:
    
    \begin{tabular}{| r || r | r | r | r | r | r | r | r | r | r | r | r |}
        \hline
        Input $x$  & 1 & 2 & 3 & 4 & 5 & 6 & 7 & 9 & 10 & 11 & 12 \\
        \hline
        Output $s$ & 0 & 1 & 1 & 3 & 1 & 6 & 1 & 4 &  7 &  1 & 16 \\
        \hline
    \end{tabular}

    \item A \emph{functional specification} written as logical propositions --- a 
    \emph{precondition} which must hold for the inputs, at the beginning of the 
    program, and a \emph{postcondition} which must hold for the outputs, at the end of 
    the program --- saying exactly what the valid inputs for the program are and 
    what output is expected for each such input. For example, the following, which 
    again specifies $s$ to end as sum of the proper divisors $k$ of $x_0$:
    
    \begin{tabular}{|ll|}
        \hline
        Pre:  & $x > 0,\ x_0 = x$ \\
        Post: & $s = \sum\{k : k|x_0,\ 0<k<x_0\}$ \\
        \hline
    \end{tabular}
    
\end{enumerate}

\clearpage
\section{Summary of Previous Work} \label{sec:prv}

As this research placement is a continuation of an undergraduate final-year individual 
project, there is an existing body of work that is being extended. This work is 
described in more detail in \cite{final}, but a summary of what was produced will be 
given here.

All Answer Set Programs referred to in this section are written for Clingo 3.0.5 
\cite{clingo3}. They may work in earlier versions of Clingo 3, but will almost
certainly not work in Clingo 4.

Files referred to in this section can be found in the project's GitHub repository at \\
\url{https://github.com/JosephCrowe/ic-while-synth/tree/master/main}.

\subsection{Program Simulator} \label{sec:sim}

An Answer Set Program, \verb|run.lp|, which computes execution traces of While 
programs, was written to serve as a component for both \ref{sec:syx} and 
\ref{sec:gnx}. Because of concerns relating to the grounding of the program, a 
modified version is used in \verb|learn.lp| for \ref{sec:syx} when synthesising 
programs (however, it is possible this could be changed to avoid that duplication). For 
details of the how \verb|run.lp| is implemented, refer to Section~4.3 of \cite{final}.

\subsubsection{Input Format} \label{sec:sim:inp}

Firstly, the simulator must be configured with the following ASP constants, to define 
execution limits:

\begin{tabularx}{\textwidth}{|l|X|}
\hline
\verb|#const time_max=TMX.| &
If more than \verb|TMX| instructions are processed in a single execution trace, that 
trace will fail and be considered to have not terminated. \\*
\hline
\verb|#const int_min=IMN.| &
If the value of any variable becomes less than \verb|IMN|, the current execution trace 
will fail. \\*
\hline
\verb|#const int_max=IMX.| &
If the value of any variables becomes greater than \verb|IMX|, the current execution 
trace will fail. \\*
\hline
\end{tabularx}

Secondly, the program itself is encoded by introducing several facts of the form
\begin{verbatim}
line_instr(N, I).
\end{verbatim}
where \verb|N| is a \emph{line number} starting at 1 and increasing by exactly 1 for
each subsequent line, and \verb|I| is an \emph{instruction}, representing a syntactic
element of the program of one of the following forms:

\begin{tabularx}{\textwidth}{|l|X|}
\hline
\verb|set(V, E)| &
Set variable \verb|V| is to the value of expression \verb|E| \\*
\hline
\verb|if(B, L)| &
Execute the next \verb|L| lines if and only if boolean guard \verb|B| holds \\*
\hline
\verb|while(B, L)| &
For as long as \verb|B| holds, repeatedly execute the next \verb|L| lines, which must 
be immediately followed by \verb|end_while|. \\*
\hline
\verb|end_while| &
May only occur \verb|L| lines after a \verb|while(B, L)| instruction. Marks the point 
where execution may return to the corresponding loop header. \\*
\hline
\end{tabularx}

\emph{Variables} are simply represented as nullary function symbols, e.g.\ \verb|x|, 
\verb|y|,\ \verb|z|,\ etc, giving the name of the variable. For the syntax of boolean 
guards and expressions, and commentary on why this representation was chosen, refer to 
Section~4.2 of \cite{final}.

Finally, each execution trace must be assigned an identifier (which may be any ASP 
term), and the initial values of relevant variables specified. If no traces are defined, 
a default trace with identifier \verb|0| is used, which starts with all variables 
uninitialised. In \verb|run.lp|, a trace is referred to as a \emph{run}, and the syntax 
for defining one is:

\begin{verbatim}
in(R, V1, C1).
in(R, V2, C2).
...
in(R, Vn, Cn).
\end{verbatim}

For the run with identifier \verb|R|, this initialises variable \verb|Vi| to the value 
\verb|Ci| (which must be an integer respecting \verb|int_min| and \verb|int_max|) for 
each \verb|i| from \verb|1| to \verb|n|.

\subsubsection{Output Format} \label{sec:sim:out}

For each variable \verb|V| and run \verb|R|, the answer set of \verb|run.lp| will 
contain a fact of one of the following forms:

\begin{tabularx}{\textwidth}{|l|X|}
\hline
\verb|run_var_out(R, V, C_final)| &
The final value of \verb|V| was the integer \verb|C_final|. \\*
\hline
\verb|run_var_out(R, V, unset)| &
\verb|V| was never set during this run. \\*
\hline
\verb|run_does_not_halt(R)| &
Due to timeout or some other error condition, the run did not successfully halt. \\*
\hline
\end{tabularx}

\subsubsection{Invocation} \label{sec:sim:inv}

To execute a simulation task, include the relevant constants and facts in an Answer Set 
Program before \verb|run.lp|. This could be done by concatenating the definitions with 
\verb|run.lp|'s text, with an \verb|#include| directive, or by giving multiple 
arguments to \verb|clingo|. An example of the second method is given below, for the 
sum-of-divisors program given earlier in this document.

\clearpage
\begin{verbatim}[samepage=true, fontsize=\footnotesize,
                 label=run\_sum\_of\_divisors.lp]
#const time_max=50.
#const int_min=0.
#const int_max=20.

line_instr(1, set(s, con(0))).               % s = 0;
line_instr(2, set(d, con(1))).               % d = 1;
line_instr(3, while(lt(var(d), var(x)), 4)). % while (d < x) {
line_instr(4, set(m, mod(var(x), var(d)))).  %     m = x % d;
line_instr(5, if(lt(var(m), con(1)), 1)).    %     if (m < 1)
line_instr(6, set(s, add(var(s), var(d)))).  %         s = s + d;
line_instr(7, set(d, add(var(d), con(1)))).  %     d = d + 1;
line_instr(8, end_while).                    % }

in(r1,x,1). in(r2,x,2). in(r3,x,3). in(r4,x,4). in(r5,x,5). in(r6,x,6).

#include "run.lp".
\end{verbatim}

Running this with \verb|clingo| produces the following output (the list of facts has 
been rearranged to be more easily readable, but is otherwise unchanged):

\begin{verbatim}[samepage=true, fontsize=\footnotesize]
$ clingo run_sum_of_divisors.lp
Answer: 1
run_var_out(r1,s,0) run_var_out(r1,x,1) run_var_out(r1,d,1) run_var_out(r1,m,unset)
run_var_out(r2,s,1) run_var_out(r2,x,2) run_var_out(r2,d,2) run_var_out(r2,m,0)
run_var_out(r3,s,1) run_var_out(r3,x,3) run_var_out(r3,d,3) run_var_out(r3,m,1)
run_var_out(r4,s,3) run_var_out(r4,x,4) run_var_out(r4,d,4) run_var_out(r4,m,1)
run_var_out(r5,s,1) run_var_out(r5,x,5) run_var_out(r5,d,5) run_var_out(r5,m,1)
run_var_out(r6,s,6) run_var_out(r6,x,6) run_var_out(r6,d,6) run_var_out(r6,m,1)
SATISFIABLE

Models      : 1     
Time        : 36.300
  Prepare   : 36.080
  Prepro.   : 0.220
  Solving   : 0.000
\end{verbatim}

\subsection{Synthesis of Programs from Examples} \label{sec:syx}

The basic form of program synthesis supported by the system takes a list of 
input/output examples, and tries to synthesise a program \emph{of a fixed length}.
This is achieved by the ASP \verb|learn.lp|. Refer to Section~5.2.1 of 
\cite{final} and to the comments in \verb|learn.lp| for more details of its 
implementation and usage.

A practical user interface is provided by \verb|IterativeLearn.hs|, which 
searches incrementally for a program of (potentially) arbitrary length by 
repeatedly invoking \verb|learn.lp|. The synthesis task is read from a 
user-written configuration file that abstracts the details of the underlying 
ASP invocation.

\subsubsection{Input Format} \label{sec:syx:inp}

The configuration file for \verb|IterativeLearn.hs| is an answer set program 
which is solved using \verb|clingo|, and the following facts in its first 
answer set examined to obtain the configuration:

\begin{tabularx}{\textwidth}{|l|l|X|}
\hline
\textbf{Fact} & \textbf{Count} & \textbf{Description} \endhead
\hline
\verb|int_range(Imin, Imax).| & $1$ &
Only programs in which variable values remain in the inclusive interval 
between \verb|Imin| and \verb|Imax| for the given examples will be considered. \\
\hline
\verb|time_limit(Tmax).| & $1$ &
Only programs which run to at most \verb|Tmax| time steps (i.e. executed 
instructions) for the given examples will be considered. \\
\hline
\verb|line_limit_min(Lmin).| & $0..1$ &
Start the incremental search at programs of length \verb|Lmin| instructions. 
Programs containing fewer instructions will not be considered. If not given, 
defaults to 0. \\
\hline
\verb|line_limit_max(Lmax).| & $0..1$ &
Stop the incremental search at programs of length \verb|Lmax| instructions.
Programs containing more instructions will not be considered. If not given, 
the search is unbounded. \\
\hline
\verb|line_limit_step(Lmin).| & $0..1$ &
Increment the program length by \verb|Lmin| upon each iteration of the search, 
meaning that only programs with $\texttt{Lmin}+k\texttt{Lstep}$ lines for some 
$k$ will be considered. If not given, defaults to 1. \\
\hline
\verb|input_variable(V).| & $0..$ &
The variable \verb|V| is initialised to an input to the program. \\
\hline
\verb|output_variable(V).| & $0..$ &
The variable \verb|V|'s final value is an output of the program. \\
\hline
\verb|in(R, V, C).| & $0..$ &
In the example identified by \verb|R|, input variable \verb|V| starts as \verb|C|. \\
\hline
\verb|out(R, V, C).| & $0..$ &
In the example identified by \verb|R|, output variable \verb|V| is required to 
end with value \verb|C| in the program. \\
\hline
\verb|constant(C).| & $0..$ &
The constant \verb|C| may occur in the program. \\
\hline
\verb|extra_variable(V).| & $0..$ &
In addition to any input and output variables, the variable \verb|V| may occur 
in the program. \\
\hline
\verb|read_only_variable(V).| & $0..$ &
The variable V may not be written to in the program. \\
\hline
\verb|preset_line_instr(L, I).| & $0..$ &
In the synthesised program, the instruction at line \verb|L| is fixed to 
\verb|I|. \verb|L| may be written in terms of the constant \verb|line_max|, 
which stands for the last line in the program. \\
\hline
\verb|disallow_feature(F).| & $0..$ &
The given language feature is prevented from occurring in the program, where the 
feature \verb|F| may be one of \verb|if|, \verb|while|, \verb|add|, \verb|sub|, 
\verb|mul|, \verb|div|, \verb|mod|, or \verb|arithmetic|, where the latter stands for 
all arithmetic operators as a shorthand. \\
\hline
\end{tabularx}

\subsubsection{Output Format} \label{sec:syx:out}

\verb|learn.lp| outputs \verb|line_instr/2| facts taking the same format as 
the input of \verb|run.lp|.

\verb|IterativeLearn.hs| prints these out in a more human-readable format 
resembling the syntax of Python (in fact, any such representation can be made 
into a working Python function with minimal changes, which is useful for 
manual verification of the program), and optionally (see below) also prints 
the ASP representation of the synthesised program.

\subsubsection{Invocation} \label{sec:syx:inv}

\verb|IterativeLearn.hs| can be compiled or run in interpreted mode using GHC 
\cite{ghc}. It \emph{should} also work with other compatible Haskell 
implementations.

On POSIX systems, it can be run directly as a shell script when 
\verb|runhaskell| is present on the system path, e.g.\ from a GHC 
installation. Below is the command-line syntax for running 
\verb|IterativeLearn.hs| as a shell script in a Mac OS, Linux or other 
POSIX-conformant terminal:

\begin{verbatim}
./IterativeLearn.hs [OPTIONS] CONFIG_FILE
\end{verbatim}

Where \verb|[OPTIONS]| may consist of zero or more of the following flags:

\begin{tabularx}{\textwidth}{|lll|X|}
\hline
\verb|--interactive| &or& \verb|-i|&
Run in interactive mode, pausing at certain points in the synthesis program 
and presenting a command-line prompt. Enter \verb|?| on the prompt to see 
further options. \\
\hline
\verb|--echo-clingo| &or& \verb|-ec| &
Show the raw output of \verb|clingo| for each invocation. \\
\hline
\verb|--echo-asp|    &or& \verb|-ea| &
Show the ASP code used with each invocation of \verb|clingo|. \\
\hline
\verb|--threads=N|   &or& \verb|-jN|&
Run up to \verb|N| instances of \verb|clingo| in parallel when there is an 
opportunity to speed up the search by doing this. \\
\hline
\end{tabularx}

\sloppy Alternatively, the user can compile it using GHC, e.g.\ 
\verb|ghc IterativeLearn.hs|, then use the name of the compiled binary,
\verb|IterativeLearn| or \verb|IterativeLearn.exe|, in place of
\verb|IterativeLearn.hs|; or directly invoke \verb|runhaskell IterativeLearn.hs|
to run it without compiling. Both of these methods should work in Windows as well
as on POSIX platforms.

Note: due to a technical detail, it is currently necessary to run 
\verb|IterativeLearn.hs| from the same directory in which it resides, when 
running it in interpreted mode.

\verb|IterativeLearn.hs| also exposes an API allowing programmatic invocation of its 
functionality via Haskell. Among others, the \verb|IterativeLearn| module contains the 
following relevant exports:

\begin{verbatim}[frame=none]
iterativeLearnConf :: Conf -> IO Program
readConfFile       :: FilePath -> Conf -> IO Conf
defaultConf        :: Conf

newtype Program = Program [LineInstr]
type LineInstr  = Clingo.Fact
\end{verbatim}

\verb|iterativeLearnConf| takes a value of type \verb|Conf|, which represents the 
ambient configuration of a synthesis task, and gives an \verb|IO| action 
either resulting in a value of type \verb|Program| or terminating the process with an 
error message.

\verb|Program| is simply a \verb|newtype| wrapper around a list of \verb|line_instr| 
facts represented as the type \verb|Fact| exported by the module \verb|Clingo|. See 
\verb|Clingo.hs| for more information about this, and \verb|While.hs| for utilities
capable of parsing such a representation.

The default configuration is given by \verb|defaultConf|, but this contains some 
undefined fields, so it is necessary to at least update it with respect to a 
configuration file, using \verb|readConfFile|.

A minimal Haskell program using this interface might be:

\begin{verbatim}
import qualified IterativeLearn as IL

main = do
    conf <- IL.readConfFile "config.lp" IL.defaultConf
    prog <- IL.iterativeLearnConf conf
    return () -- Do nothing with the generated program.
\end{verbatim}

It is acknowledged that an interface with fewer side-effects, perhaps using a function 
\verb|:: Conf -> IO (Maybe Program)| combined with a configuration option to 
suppress the default textual output, would be more useful; but this does not currently 
exist.

\subsubsection{Example Usage} \label{sec:syx:exu}

Below is a configuration file written for \verb|IterativeLearn| specifying a synthesis 
task for part of the sum-of-divisors program, where the remaining parts are given by 
the user:

\begin{verbatim}[samepage=true, fontsize=\footnotesize, label=sum\_of\_divisors\_ex.lp]
int_range(0, 10).
time_limit(40).
line_limit_min(4).

constant(1).
disallow_feature(sub; div; mul; while).

input_variable(x).
output_variable(s).
read_only_variable(x; d).
extra_variable(d; m).

 in(r1,x,1).  in(r2,x,2).  in(r3,x,3).  in(r4,x,4).  in(r5,x,5).  in(r6,x,6).
out(r1,s,0). out(r2,s,1). out(r3,s,1). out(r4,s,3). out(r5,s,1). out(r6,s,6).

preset_line_instr(1, set(s, con(0))).
preset_line_instr(2, set(d, con(1))).
preset_line_instr(3, while(lt(var(d), var(x)), line_max-4)).
preset_line_instr(line_max-1, set(d, add(var(d), con(1)))).
preset_line_instr(line_max, end_while).
\end{verbatim}

This specification is executed with \verb|IterativeLearn| on a machine with 4 CPU 
cores, so we allow 4 concurrent threads. We also enable the display of Clingo's output 
so we can see progress during concurrent batches (some lines are rearranged to fit on 
this page):

\begin{verbatim}[samepage=true, fontsize=\footnotesize]
$ ./IterativeLearn.hs --threads=4 --echo-clingo sum_of_divisors_ex.lp
Searching for a program with 4-7 lines satisfying 6 example(s)...
4-> UNSATISFIABLE
5-> UNSATISFIABLE
6-> UNSATISFIABLE
7-> UNSATISFIABLE
No such program found.
Searching for a program with 8-11 lines satisfying 6 example(s)...
8-> Answer: 1
8-> line_instr(1,set(s,con(0)))              line_instr(5,if(lt(var(m),con(1)),1))
    line_instr(2,set(d,con(1)))              line_instr(6,set(s,add(var(d),var(s))))
    line_instr(3,while(lt(var(d),var(x)),4)) line_instr(7,set(d,add(var(d),con(1))))
    line_instr(4,set(m,mod(var(x),var(d))))  line_instr(8,end_while)
8-> SATISFIABLE
Found the following program:
   1. s = 0
   2. d = 1
   3. while (d < x):
   4.     m = x % d
   5.     if (m < 1):
   6.         s = d + s
   7.     d = d + 1
   8. end_while
\end{verbatim}

This example is used to showcase many of the features of the synthesiser, but 
it does rely on the user specifying in advance several of the program's lines, 
including its looping structure, since without this the problem could not be solved in 
a reasonable time. These restrictions will be partially lifted when we use more 
advanced ideas, described later.

\subsection{Automatic Generation of Examples} \label{sec:gnx}

The extended functionality of \verb|IterativeLearn| includes the ability to take as input a 
precondition and postcondition, and generate as an intermediate stage the (minimal) set of examples 
needed to synthesise a program \emph{correct} for all inputs lying in the configured domain.

More specifically, a program is defined for this purpose to be \emph{correct} with respect to a 
precondition $P$ defined over the initial values of input variables, a postcondition $Q$ defined 
over the initial values of input variables and the final values of output variables, a domain 
$D\subseteq\mathbb{Z}$ and a time limit $T_\mathrm{max}$, iff for all combinations of inputs lying 
in $D$ for which $P$ holds, the program terminates with $T_\mathrm{max}$ time steps or fewer and 
$Q$ holds for its initial and final states.

This functionality is achieved using a new component, the counterexample-finder, which consists 
partially of dynamically generated ASP derived from the pre- and postcondition, and partially of 
the ASP \verb|counterexample.lp|.

\subsubsection{Input Format} \label{sec:gnx:inp}

The input format is the same as for the basic form of \verb|IterativeLearn| (see 
Section~\ref{sec:syx:inp}), with the addition of the following configuration entries:

\begin{tabularx}{\textwidth}{|l|l|X|}
\hline
\textbf{Fact} & \textbf{Count} & \textbf{Description} \endhead
\hline
\verb|precondition(P).| & $0..1$ &
The precondition to use when synthesising the program, encoded as an ASP string. If not given, 
defaults to the empty condition (always true). \\*
\hline
\verb|postcondition(Q).| & $0..1$ &
The postcondition to use when synthesising the program, encoded as an ASP string. If not given, 
defaults to the empty condition (always true). \\*
\hline
\end{tabularx}

In the above, the encoding of \verb|P| and \verb|Q| is (currently) the same as the body of an ASP rule in Clingo 3 (excluding the terminating 
\verb|.| character), where the first letter of each variable has been made uppercase, and therefore the variables occur as free ASP variables 
in the condition. In order to express disjunction and complicated arithmetical functions, conditions may refer to constants and predicates 
defined in the specification file, as long as the names of these do not coincide with any used internally by \verb|learn.lp| or 
\verb|counterexample.lp|. See Section~\ref{sec:gnx:exu} below for an example.

Of course, it is not ideal to have to impose this requirement on users of the system, so this will probably be changed in the future.

\subsubsection{Output Format} \label{sec:gnx:out}

The output format is the same as for the basic form of \verb|IterativeLearn| (see Section~\ref{sec:syx:out}), taking the form of a 
real-time account of the synthesiser's progress, but with the addition of information about the newly introduced features. In particular, it
includes the intermediate programs, and information about each counterexample generated by the system.

\subsubsection{Invocation} \label{sec:gnx:inv}

The command-line syntax for invoking \verb|IterativeLearn| is unchanged with this extension: see Section~\ref{sec:syx:inv}. The programmatic 
interface is also the same (see Section~\ref{sec:syx:inv}), except of course that the configuration \verb|:: Conf| supplied to 
\verb|iterativeLearnConf| must contain a non-empty postcondition in order for the new functionality to apply.

\subsubsection{Example Usage} \label{sec:gnx:exu}

Below is a configuration file specifying the sum-of-divisors program as in 
Section~\ref{sec:syx:exu}, but using a precondition and a postcondition instead of 
explicitly listed examples. The postcondition depends on some further predicates 
defined in the same file.

\begin{verbatim}[fontsize=\footnotesize, label=sum\_of\_divisors\_cond.lp]
int_range(0, 6).
time_limit(40).
line_limit_min(4).

constant(1).
disallow_feature(sub; div; mul; while).

input_variable(x).
output_variable(s).
read_only_variable(x; d).
extra_variable(d; m).

precondition("In_x > 0").
postcondition("divisor_sum(In_x, Out_s)").

divisor_sum(X, D) :-
    partial_divisor_sum(X, X, D).

partial_divisor_sum(1, X, 0) :-
    X=1..10.
partial_divisor_sum(I+1, X, D+I) :-
    X=1..10, I < X, X #mod I == 0,
    partial_divisor_sum(I, X, D).
partial_divisor_sum(I+1, X, D) :-
    X=1..10, I < X, X #mod I != 0,
    partial_divisor_sum(I, X, D).

preset_line_instr(1, set(s, con(0))).
preset_line_instr(2, set(d, con(1))).
preset_line_instr(3, while(lt(var(d), var(x)), line_max-4)).
preset_line_instr(line_max-1, set(d, add(var(d), con(1)))).
preset_line_instr(line_max, end_while).
\end{verbatim}

This yields the following output:

\begin{verbatim}[fontsize=\footnotesize]
./IterativeLearn.hs --threads=4 sum_of_divisors_cond.lp 
Searching for a program with 4-7 lines satisfying 0 example(s)...
Found the following program:
   1. s = 0
   2. d = 1
   3. while (d < x):
   4.     d = d + 1
   5. end_while
Searching for a counterexample to falsify the postcondition...
Found the following counterexample:
   Input:    x=2
   Expected: s=1
   Output:   s=0
Searching for a program with 5-8 lines satisfying 1 example(s)...
Found the following program:
   1. s = 0
   2. d = 1
   3. while (d < x):
   4.     s = 1 % x
   5.     d = d + 1
   6. end_while
Searching for a counterexample to falsify the postcondition...
Found the following counterexample:
   Input:    x=4
   Expected: s=3
   Output:   s=1
Searching for a program with 6-9 lines satisfying 2 example(s)...
Found the following program:
   1. s = 0
   2. d = 1
   3. while (d < x):
   4.     s = d
   5.     d = d + 1
   6. end_while
Searching for a counterexample to falsify the postcondition...
Found the following counterexample:
   Input:    x=5
   Expected: s=1
   Output:   s=4
Searching for a program with 6-9 lines satisfying 3 example(s)...
Found the following program:
   1. s = 0
   2. d = 1
   3. while (d < x):
   4.     if (s < 1):
   5.         s = d + d
   6.     s = d % s
   7.     d = d + 1
   8. end_while
Searching for a counterexample to falsify the postcondition...
Found the following counterexample:
   Input:    x=6
   Expected: s=6
   Output:   s=0
Searching for a program with 8-11 lines satisfying 4 example(s)...
Found the following program:
   1. s = 0
   2. d = 1
   3. while (d < x):
   4.     m = x % d
   5.     if (m < 1):
   6.         s = d + s
   7.     d = d + 1
   8. end_while
Searching for a counterexample to falsify the postcondition...
Success: the postcondition could not be falsified.
\end{verbatim}

\subsubsection{Implementation} \label{sec:gnx:imp}

\clearpage
\section{Current Extensions} \label{sec:cur}

\clearpage
\begin{thebibliography}{9}
    \bibitem{final}
        Synthesis of Simple While Programs using Answer Set Programming (Final Report). 
        \\ \url{http://www.doc.ic.ac.uk/~jjc311/while-synth/WS_FinalReport.pdf}
    \bibitem{clingo3}
        Clingo 3.0.5. \emph{The Potsdam Answer Set Solving Collection.}
        \\ \url{http://sourceforge.net/projects/potassco/files/clingo/3.0.5}
    \bibitem{ghc}
        The Glasgow Haskell Compiler.
        \\ \url{https://www.haskell.org/ghc}
\end{thebibliography}

\end{document}
